\documentclass[dutch]{ltugboat}
\usepackage{babel}
\usepackage[T1]{fontenc}
\usepackage{graphicx}
\usepackage{microtype}
\usepackage[hidelinks,pdfa]{hyperref}

\title{De Auteur}

% repeat info for each author; comment out items that don't apply.
\author{Erik Nijenhuis}
\address{Frans Halsstraat 38\\ Leeuwarden, 8932 JC \\ The Netherlands}
\netaddress{erik (at) xerdi dot com}
\personalURL{https://www.xerdi.com}
%\ORCID{0}
% To receive a physical copy of the TUGboat issue, please include the
% mailing address we should use, as a comment if you prefer it not be printed.

\begin{document}
    \maketitle

    Mijn \LaTeX\ avontuur als gebruiker is begonnen in 2016 en sinds eind 2023 is mijn avontuur begonnen als \LaTeX-pakket auteur.
    Ik heb \LaTeX\ ingezet voor wetenschappelijke-, kantoor- en juridische doeleinden.
    Voor de wetenschappelijke doeleinden heb ik voornamelijk technische documentatie geschreven specifiek voor software-projecten, zoals technisch ontwerp, functioneel ontwerp, of zelfs onderzoeksplannen en -rapporten.
    Voor kantoor doeleinden heb ik een vrije software-applicatie geschreven voor GNOME Linux systemen, genaamd GinVoice, die onder water \LaTeX\ gebruikt om facturen te kunnen maken.
    Afgelopen- en komende tijd blijf ik mezelf toewijden aan de missie om \LaTeX\ geschikt te maken voor Nederlandse juristen en ik hoop daarmee het draagvlak van \LaTeX\ voor die branche te vergroten, zodat voor mijn bedrijf veel werk ontstaat op het gebied van systeemintegratie en opleiding.

    \paragraph{Technische Auteur}
    Mijn \LaTeX\ avontuur als gebruiker is begonnen tijdens mijn studie HBO-ICT Informatica aan de Hogeschool Windesheim te Zwolle.
    Ik heb me tijdens die studie voorgenomen alle stof wat te maken had met closed-source oplossingen, zoals MS Windows 10, te vervangen met vrije software-alternatieven, zoals Arch-Linux of Debian Linux.
    Helaas was dat niet altijd even handig, want praktisch gezien was ik ieder semester weer druk met het installeren van Windows om de toets te kunnen maken.
    Door deze koppige houding, en dankzij Joram Schrijver die me initieel in aanraking bracht met \LaTeX\ tijdens een schoolproject waarin we een cross-platform computerspel ontwikkelde, ben ik \LaTeX\ gaan zien als de vrije documentatie-tooling voor Software Engineers zoals ikzelf.
    Alleen al vanwege de tekstuele invoerbron is het voor ontwikkelaars geschikt, omdat ze versiebeheer kunnen inschakelen, net als voor iedere andere programmeertaal.
    Ik ben toen zelfs zo brutaal geweest een onderzoek te starten voor de hogeschool om te kijken of \LaTeX\ in combinatie met \texttt{Git} een geschikte oplossing zou zijn voor al het zelf ontwikkelde onderwijsmateriaal van HBO-ICT\@.
    Deze combinatie bleek uitermate geschikt te zijn voor het samenwerken aan \LaTeX-projecten en de ondervraagde docenten zagen wel degelijk iets in de uiteindelijk aangedragen werkwijze.
    Echter, is zoiets vakkundigs lastig door te voeren in zo'n grote organisatie.
    Daarnaast was de Hogeschool uitsluitend geïnteresseerd in MS Office oplossingen, \cs{bye}.

    \paragraph{\LaTeX-pakket auteur}
    In april 2022 kwam ik in aanmerking met Jarno Koning van VeiligDoen B.V.\ voor het opstellen van voorwaarden en contracten voor mijn eigen onderneming.
    Uiteraard was ik mijn koppigheid niet verloren en verwerkte ik zelf de juridische inhoud in \LaTeX-documenten aan de hand van een aangeleverde MS Word document.
    Bij ieder document wat ik heb laten opstellen heb ik meerdere malen demonstraties en aanbevelingen gegeven over hoe de juridische inhoud technisch het beste verwezenlijkt kan worden in \LaTeX.

    Na aanleiding van deze gesprekken en enige belangstelling daarvoor te hebben bespeurd, heb ik radicaal besloten om de oplossing voor juristen onder de LPPLv1.3c te gaan verpakken en te publiceren, zodat de auteursgereedschappen voor juristen standaard aanwezig zijn in \TeX\ Live.
    Voor mij als eenmanszaak was dit een flinke diepte-investering.
    Ik kende de gemeenschappen, destijds nog als \LaTeX-gebruiker, nauwelijks en \LaTeX-pakketten maken is geen koud kunstje kwam ik al snel achter.
    Daarnaast zag ik op tegen het betreden van die gemeenschappen vanwege het gemiddelde academische niveau.

    Pas in november 2023 sloeg ik mijn slag en uploade ik mijn eerste \LaTeX-pakket — \texttt{gitinfo-lua}.
    Ik kwam voor het eerst in aanmerking met \CTAN\ en het begon vrij stroef.
    Ik kreeg veel aanmerkingen op pakketnaam, bestandsnaamgeving, mappenstructuur, uploadformaten, et cetera.
    Uiteindelijk ben ik, dankzij Petra Rübe-Pugliese van \CTAN\ en Karl Berry van \TeX\ Live, een stuk wijzer geworden over de TDS en pakketbeheer/ontwikkeling en ging het, pakket na pakket, steeds makkelijker om gepubliceerd te krijgen.

    Echter, mijn reis als \LaTeX-pakket auteur is hier niet geëindigd.
    Dankzij Karl zijn alertheid ben ik in contact gekomen met Dr Nicola L C Talbot om de Nederlandse taaldefinitie in \texttt{fmtcount} verpakt te krijgen.
    Deze definitie wordt nu slinks meegeleverd via pakket \texttt{regulatory}, wat naar mijn idee een zeer terechte aanmerking is van Karl.
    Sindsdien ben ik totaal anders gaan kijken naar de gemeenschappen, waar ik tot eerder nog zo tegen op keek.
    Ik zie het nu als een hele hartelijke groep van vrijwilligers, die net als ik maar al te graag wat vertellen over de wondere wereld van \LaTeX.

    In plaats van terug te deinzen zoals ik aanvankelijk deed, moedig ik iedereen aan om het diepe in te springen en de wereld van \LaTeX-pakketontwikkeling te verkennen.
    Het delen van kennis en bijdragen aan de \LaTeX-gemeenschap is een verrijkende ervaring die niet alleen de software verbetert, maar ook een gevoel van voldoening en verbondenheid biedt.
    Laten we samen bouwen aan een nog sterker \LaTeX-ecosysteem!
\end{document}
