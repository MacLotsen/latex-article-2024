\documentclass[english]{ltugboat}
\usepackage{babel}
\usepackage[T1]{fontenc}
\usepackage{graphicx}
\usepackage{microtype}
\usepackage[hidelinks,pdfa]{hyperref}

\title{The Author}

% repeat info for each author; comment out items that don't apply.
\author{Erik Nijenhuis}
\address{Frans Halsstraat 38\\ Leeuwarden, 8932 JC \\ The Netherlands}
\netaddress{erik (at) xerdi dot com}
\personalURL{https://www.xerdi.com}
%\ORCID{0}
% To receive a physical copy of the TUGboat issue, please include the
% mailing address we should use, as a comment if you prefer it not be printed.

\begin{document}
    \maketitle

    My \LaTeX\ journey as a user began in 2016, and since late 2023, my adventure as a \LaTeX\ package author began.
    I have used \LaTeX\ for scientific, office, and legal purposes.
    For scientific purposes, I have mainly written technical documentation specifically for software projects, such as technical designs, functional designs, or even research plans and reports.
    For office purposes, I developed a free software application for GNOME Linux systems called GinVoice, which uses \LaTeX\ under the hood to generate invoices.
    In the past and upcoming times, I am dedicated to the mission of making \LaTeX\ suitable for Dutch lawyers, hoping to increase the adoption of \LaTeX\ in that industry, thereby generating significant work for my company in terms of system integration and training.

    \paragraph{Technical Writer}
    My journey with \LaTeX\ as a user began during my study of HBO-ICT Computer Science at Hogeschool Windesheim in Zwolle.
    During that study, I decided to replace all materials related to closed-source solutions, such as MS Windows 10, with free software alternatives, such as Arch Linux or Debian Linux.
    Unfortunately, this was not always practical, as practically speaking, I was busy reinstalling Windows every semester to take the exams.
    Due to this stubborn attitude, and thanks to Joram Schrijver who initially introduced me to \LaTeX\ during a school project where we developed a cross-platform computer game, I began to see \LaTeX\ as the free documentation tooling for Software Engineers like myself.
    Simply because of the textual input source, it is suitable for developers, as they can enable version control, just like any other programming language.
    I even had the audacity to start a study for the university to see if \LaTeX\ in combination with \texttt{Git} would be a suitable solution for all the self-developed educational materials of HBO-ICT\@.
    This combination turned out to be extremely suitable for collaborating on \LaTeX\ projects, and the surveyed teachers did see merit in the proposed workflow.
    However, implementing something so professional was difficult in such a large organization.
    Additionally, the University was only interested in MS Office solutions, \cs{bye}.

    \paragraph{\LaTeX\ Package Author}
    In April 2022, I came into contact with Jarno Koning of VeiligDoen B.V.\ for drafting terms and contracts for my own company.
    Of course, I hadn't lost my stubbornness and processed the legal content myself in \LaTeX\ documents based on a provided MS Word document.
    For each document I had drafted, I gave multiple demonstrations and recommendations on how the legal content could be best realized in \LaTeX.

    Following these discussions and having noticed some interest in it, I decided radically to package and publish the solution for lawyers under the LPPLv1.3c, so that the authoring tools for lawyers are standard included in \TeX\ Live.
    For me as a sole proprietor, this was a significant investment.
    I hardly knew the communities, still as a \LaTeX\ user, and making \LaTeX\ packages turned out to be no easy task, I quickly found out.
    Additionally, I was hesitant to enter those communities because of the average academic level.

    It wasn't until November 2023 that I made my move and uploaded my first \LaTeX\ package — \texttt{gitinfo-lua}.
    This was my first encounter with \CTAN\ and it started quite rough.
    I received many comments on package name, file naming, directory structure, upload formats, etc.
    Ultimately, thanks to Petra Rübe-Pugliese of \CTAN\ and Karl Berry of \TeX\ Live, I became much wiser about the TDS and package management/development, and it became easier and easier to get published, package after package.

    However, my journey as a \LaTeX\ package author did not end here.
    Thanks to Karl's alertness, I got in touch with Dr. Nicola L C Talbot to get the Dutch language definition packaged in \texttt{fmtcount}.
    This definition is now cleverly included via the \texttt{regulatory} package, which I think is a very valid point by Karl.
    Since then, I have been looking at the communities completely differently, where I once looked so unfavorably.
    I now see it as a very welcoming group of volunteers, who, just like me, are more than happy to tell you all about the wonderful world of \LaTeX.

    Instead of shying away as I initially did, I encourage everyone to dive in and explore the world of \LaTeX\ package development.
    Sharing knowledge and contributing to the \LaTeX\ community is an enriching experience that not only improves the software but also provides a sense of fulfillment and camaraderie.
    Let's build an even stronger \LaTeX\ ecosystem together!
\end{document}
