\documentclass[11pt]{article}


\begin{document}
    \noindent
    Mogelijke onderwerpen voor artikelen per \LaTeX-pakket:

    \begin{description}
        \item[gitinfo-lua] Ontwikkelt met Lua\LaTeX{}.\\
        Mogelijke onderwerpen:
        \begin{itemize}
            \item Samenwerken in een \LaTeX-project met Git.\\
            {
                \footnotesize
                Niet per se gerelateerd aan \texttt{gitinfo-lua}, echter weet ik niet hoe bekend Git is voor het publiek van MAPS.
                Ik zou een kort artikel of een paragraaf daarover kunnen schrijven, mocht daar behoefte voor zijn.
                Aan de andere kant staat het internet vol met voorbeelden van Git.\\
                Sleutelwoorden: \textbf{git, samenwerken, wijzigingsbeheer}
            }
            \item Documentversies en -auteurs bepalen vanuit Git.\\
            {
                \footnotesize
                Basis functionaliteit van \texttt{gitinfo-lua} die uitsluitend vertrouwt op data in Git.
                Dit verhuist de administratie van auteurs en documentversies naar Git en zorgt er dus voor dat er niet meer in \LaTeX-bestanden versies of auteurs aangepast hoeven te worden.
                Kan eventueel aangevuld worden met wat Git handson.\\
                Sleutelwoorden: \textbf{versiebeleid, auteursbeleid, eenduidig beleid}
            }
            \item Documentwijzigingen opsommen vanuit Git.\\
            {
                \footnotesize
                De meest geävanceerde functionaliteit die het pakket biedt, namelijk het opsommen van de Git historie in bijvoorbeeld een \texttt{tabular} omgeving.\\
                Sleutelwoorden: \textbf{wijzigingstabel}
            }
        \end{itemize}
        \item[lua-placeholders] Ontwikkelt met Lua\LaTeX{}.\\
        Mogelijke onderwerp:
        \begin{itemize}
            \item Documentvariabelen specificeren met YAML.\\
            {
                \footnotesize
                Dit pakket heb ik voornamelijk geschreven om data buiten het \LaTeX\ domein te houden, wat het perfect maakt voor documentautomatisering, waar ik als Software Engineer veel mee te maken heb.
                Voor een probleemstelling zou ik de source van GinVoice erbij kunnen pakken, want daar is het nog provisorisch in \LaTeX\ en Python verwikkeld.
                Hierin kan ik laten zien hoe los van \LaTeX-bestanden documentvoorbeelden en ingevulde versies gecompileerd kunnen worden.
                Zonder het aanleveren van een extra YAML-bestand worden ontbrekende variabelen ingevuld met een placeholder.
                Het idee daarachter is dat een \LaTeX-bestand ten alle tijde gecompileerd kan worden.\\
                Sleutelwoorden: \textbf{YAML, data, systeemintegratie}
            }
        \end{itemize}
        \item[regulatory] Ontwikkelt met \LaTeX{}.\\
        Mogelijke onderwerpen:
        \begin{itemize}
            \item Juridische structuren.\\
            {
                \footnotesize Korte demonstratie van nieuwe structuren, zoals artikelen, leden en onderdelen.\\
                Sleutelwoorden: \textbf{structuren, ondergetekenden, algemene bepalingen, handtekeningvelden}
            }
            \item Definitielijsten met \texttt{bib2gls}.\\
            {
                \footnotesize Definities worden aangehaald met \verb|\gls| en gespecificeerd in bib-bestanden.
                Deze zijn ook aan te halen vanuit andere documenten.\\
                Sleutelwoorden: \textbf{definities, bib2gls, glossaries}
            }
            \item Automatisch verwijzingen formuleren.\\
            {
                \footnotesize Het verwijzingssysteem van regulatory kan a.d.h.v.\ de nieuwe structuren een juridische verwijzing automatisch formuleren in Engels en Nederlands.
                De verwijzingen, zowel externe verwijzingen, zijn voorzien van hyperlinks.\\
                Sleutelwoorden: \textbf{hyperref, zref}
            }
            \item Documenten als bijlage toevoegen.\\
            {
                \footnotesize Voor extra volledigheid is deze functionaliteit toegevoegd.
                Namelijk, voor een overeenkomst is het handig als de algemene voorwaarden daaraan toegevoegd is, zodat het sowieso duidelijk is welke versie van de algemene voorwaarden destijds van toepassing waren.
                Dit pakket speelt daarnaast slim in bij de eerste verschijning van een externe verwijzing en plaatst daarbij een corresponderende footnote.\\
                Sleutelwoorden: \textbf{attachfile2, footnotes}
            }
        \end{itemize}
    \end{description}

\end{document}