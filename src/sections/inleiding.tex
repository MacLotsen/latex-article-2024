\section{Introduction}
During my work as a Software Engineer, I encountered a challenge for a company that drafts agreements and terms for multiple clients.
One of the challenging aspects was keeping client data and regulatory documentation separate.
Previously, I addressed this challenge in GinVoice\cite{ginvoice} by generating additional \LaTeX\ files with Python, which were then compiled alongside the main \LaTeX\ file.
However, this time, my goal was to provide a solution from within the \LaTeX\ domain itself, rather than the application domain.
The solution I developed, now known as \pkg{lua-placeholders}\cite{lua-placeholders}, introduces a shared data layer with YAML between \LaTeX\ and application code.
The package provides an intermediary layer specifically for data through YAML files.
To demonstrate this solution, we take GinVoice as an example.
This example, a Python GTK application that generates invoices with \LaTeX, offers slightly more complexity and challenges than the legal domain has to offer.

\subsection{The Compiler -- \LuaLaTeX}
My decision to use \LuaLaTeX\ as the compiler had several reasons.
Since 2016, I have been using \LuaLaTeX, which greatly helped me with documents within computer science at the time.
Over the years, I have gained a lot of experience in compiling with \LuaLaTeX\ and see it as a suitable compiler as a developer, thanks to the ability to script in Lua, which I naturally appreciate as a programmer.

The ability to script in Lua offers several advantages.
It allows me to perform complex tasks during the compilation process, such as processing YAML files or manipulating and structuring data. Additionally, \LuaLaTeX\ supports Lua init scripts, allowing me to implement a custom compilation process with its own CLI (Command Line Interface), further simplifying and optimizing the integration process for end solutions.

\subsection{What is YAML?}
As a DevOps Engineer, I have often encountered YAML while working with tools such as Docker Compose, Travis CI, GitHub Actions, and Canonical's NetPlan (Ubuntu systems).
YAML is widely used in the DevOps world for automating and managing configurations, functioning as a structured markup language for defining configuration files and capturing infrastructural and operational aspects of software applications.

YAML has become a crucial component of modern software development and deployment due to its simple syntax and flexibility. In combination with \LaTeX, YAML provides a powerful mechanism for defining and managing structured data, which is particularly useful when integrating client data into \LaTeX\ documents. Listing~\ref{lst:example} shows an example of YAML used in conjunction with \LaTeX.

\begin{lstlisting}[language=YAML,caption={\ttfamily invoice-001.yaml},label={lst:example}]
supplier: grapefruit
client: juicing-joker
title: Grapefruit Inc. Invoice
subtitle: for fruits and stuff
currency: \$
number: 1
date: \today
...
\end{lstlisting}
