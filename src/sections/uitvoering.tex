\section{Uitvoering}\label{sec:output}
% Geavanceerde YAML payload voorbeelden, + sequence diagram command line
Nu de legacy factuur volledig is getransformeerd, kunnen we gaan kijken wat het resultaat daarvan is.
Mocht je mee willen doen via de commandline, bekijk dan de volledige source-code\cite{ginvoice-template} van deze voorbeelden.

\subsection{De Template Versie}
Zonder enige waarden aan te leveren krijgen we het volgende resultaat, te zien in figuur~\ref{fig:template}.\\
\setlength\fboxsep{0pt}%
\begin{figure}[!ht]%
    \fbox{\includegraphics[width=\linewidth-1pt]{invoice-template.pdf}}%
    \caption{\texttt{invoice-template.pdf}}\label{fig:template}%
\end{figure}
Zoals eerder gezegd is \pkg{lua-placeholders} alleen met \LuaLaTeX\ te compileren.
Daarnaast is de optie \texttt{--shell-escape} vereist, om YAML-bestanden in te kunnen laden.
Het voorbeeld kan als volgt gecompileerd worden:
\begin{lstlisting}[language=bash,caption={Compileren met \texttt{lualatex}}]
lualatex --shell-escape \
     --jobname=invoice-template \
     --output-directory=$(OUTPUT_DIR) \
       invoice
\end{lstlisting}
Waar \texttt{\$(OUTPUT\_DIR)} de gewenste uitvoer map is.

Echter, als je een template aan het opmaken bent, dan is blijvend genereren met \LaTeX MK gebruiksvriendelijker:
\begin{lstlisting}[language=bash,caption={Compileren met \texttt{latexmk}}]
latexmk -pvc -lualatex \
    --shell-escape \
    --jobname=invoice-template \
    --output-directory=$(OUTPUT_DIR) \
      invoice
\end{lstlisting}
Zo hoef je bij iedere wijziging in \TeX\ niet opnieuw te compileren, maar gebeurt dat vanzelf.

\subsection{YAML-waarden}
Om een ingevulde factuur te krijgen zullen we de volgende YAML-bestanden nodig hebben:

\dirtree{%
    .1 \meta{project dir}.
    .2 invoices.
    .3 \meta{invoice}.yaml.
    .2 suppliers.
    .3 \meta{supplier}.yaml.
    .2 styles.
    .3 \meta{style}.yaml.
    .2 clients.
    .3 \meta{client}.yaml.
}

Deze structuur is gebaseerd op de implementatie beschreven in hoofdstuk~\ref{sec:preamble}.
Voordat de inhoud van de YAML-bestanden besproken wordt, kunnen we eerst alternatieve projectstructuren overwegen.

\subsubsection{Alternatieve Projectstructuur}
Het staat iedereen natuurlijk vrij om zelf een gewenste mappenstructuur aan te leggen.
Zo zou je bijvoorbeeld styles onder\\
\hspace*{4em}\texttt{/suppliers/\meta{supplier}/style.yaml}\\
kunnen zetten, zodat je zelfs het \texttt{style} veld kunt weglaten in het leveranciers-recept.
Een ander goed voorbeeld is bijvoorbeeld de \texttt{clients} map onder leveranciersniveau kunnen plaatsen, zodat je niet per ongeluk klanten van leveranciers door elkaar haalt.
Dit zou je kunnen realiseren door het volgende te doen:
\dirtree{%
    .1 \meta{project dir}.
    .2 suppliers.
    .3 \meta{supplier}.yaml.
    .3 \meta{supplier}.
    .4 \meta{client}.yaml.
}
\noindent
Op deze manier zou de implementatie voor het inladen van klanten de variabelen \meta{supplier} en \meta{client} vereisen, om vervolgens tot het pad\\
\hspace*{4em}\texttt{suppliers/\meta{supplier}/\meta{client}.yaml}\\
te komen.

Dezelfde overweging zou voor de facturen toegepast kunnen worden, echter is dit een lastiger scenario, aangezien de factuurdata wordt gebaseerd op de \cs{jobname} in de implementatie van hoofdstuk~\ref{sec:preamble}.
Een mogelijke oplossing hiervoor is om het project per leverancier te beheren.
Je kunt dan de \textit{recepten} in de map \texttt{\$TEXMFHOME/tex} zetten, zodat deze voor alle projecten aanwezig zijn.
Hier een voorbeeld van een mogelijke projectstructuur:\\
\begin{minipage}{.49\linewidth}
\dirtree{%
    .1 \$HOME/texmf/tex.
    .2 recipes.
    .3 invoice.yaml.
    .3 client.yaml.
    .3 supplier.yaml.
    .3 style.yaml.
    .2 invoice.cls.
    .2 invoice.tex.
}
\end{minipage}%
\hfill
\begin{minipage}{.49\linewidth}
\dirtree{%
    .1 \meta{project dir}.
    .2 invoices.
    .3 \meta{invoice}.yaml.
    .2 clients.
    .3 \meta{client}.yaml.
    .2 supplier.yaml.
    .2 style.yaml.
}
\end{minipage}
In dit voorbeeld is alle data met betrekking op een leverancier gescheiden, waaronder de klantinformatie en de uiteindelijke facturen.

\subsection{Leveranciers en Klanten}\label{sec:supplier and client}
In het voorbeeld resultaat van GinVoice was een klant \textit{Juicing Joker} te zien.
In YAML zou dat neer komen op:
\lstinputlisting[captionpos=b,caption={\texttt{clients/juicing-joker.yaml}}]{demo/clients/juicing-joker.yaml}

Op deze manier kan de klant aangehaald worden in de factuur met \texttt{juicing-joker}.

Voor de leverancier zagen we \textit{Grapefruit Inc.}\ in het voorbeeld, wat neerkomt op:
\lstinputlisting[captionpos=b,caption={\texttt{suppliers/grapefruit.yaml} of \texttt{grapefruit/supplier.yaml}}]{demo/suppliers/grapefruit.yaml}

En voor de stijl:
\lstinputlisting[captionpos=b,caption={\texttt{styles/grapefruit.yaml} of \texttt{grapefruit/style.yaml}}]{demo/styles/grapefruit.yaml}

Het voordeel van de alternatieve projectstructuur is dat \texttt{invoice-template} automatisch de styling oppakt evenals de leveranciersinformatie, te zien in figuur~\ref{fig:result alt}.

\onecolumn
\subsection{Facturen}
\begin{figure}[!ht]
%    \centering
    \begin{subfigure}{.32\linewidth}
        \fbox{\includegraphics[width=\linewidth-1pt]{invoice2-template.pdf}}
        \caption{\texttt{invoice-template.pdf}}\label{fig:result alt}
    \end{subfigure}\hfill
    \begin{subfigure}{.32\linewidth}
        \fbox{\includegraphics[width=\linewidth-1pt]{invoice-001.pdf}}
        \caption{\texttt{invoice-001.pdf}}\label{fig:result 1}
    \end{subfigure}\hfill
    \begin{subfigure}{.32\linewidth}
        \fbox{\includegraphics[width=\linewidth-1pt]{invoice-002.pdf}}
        \caption{\texttt{invoice-002.pdf}}\label{fig:result 2}
    \end{subfigure}
    \caption{Factuurvoorbeelden}\label{fig:pdfs}
\end{figure}
\begin{multicols}{2}
    \noindent
    Om een factuur te maken die exact overeenkomt met het standaard voorbeeld van GinVoice, zoals te zien in figuur~\ref{fig:result 1}, maken we gebruik van het volgende YAML-voorbeeld:
    \lstinputlisting[xleftmargin=15pt,numbers=left,linerange=-15,caption={\texttt{invoices/invoice-001.yaml}}]{demo/invoices/invoice-001.yaml}
    De actoren \texttt{grapefruit} en \texttt{juicing-joker}, besproken in hoofdstuk~\ref{sec:supplier and client}, zien we terug in de factuur.
    Daarnaast heeft het voorbeeld dezelfde algemene informatie om tot hetzelfde resultaat te komen.
    Bij het veld \texttt{records} zie je dat een rij van de tabel veel regels inneemt.
    Bij de tweede rij van de tabel zie je dat het veld \texttt{description} een lege waarde heeft.
    \columnbreak
    Mochten de quotes weggelaten worden in YAML, dan krijg je een fout bij het converteren naar data.
    Aangezien de rijen niet al te veel verschillen van elkaar, vervolgen we het voorbeeld bij het veld \texttt{totals}:
    \lstinputlisting[xleftmargin=15pt,numbers=left,linerange=34-,firstnumber=34]{demo/invoices/invoice-001.yaml}
    Als laatste in het voorbeeld zien we de totalen en de slottekst.

    Deze factuur kan vervolgens gecompileerd worden met de volgende opdracht:
    \begin{lstlisting}[language=bash]
lualatex --shell-escape \
    --jobname=invoice-001 \
    --output-directory=$(OUTPUT_DIR) \
      invoice
    \end{lstlisting}
%    Het enige verschil tussen het compileren van de template versie is de \cs{jobname}.
\end{multicols}
\twocolumn

%\begin{figure}[!ht]
%    \centering
%    \includegraphics[width=.8\linewidth]{invoice-002.pdf}%
%    \makebox[0pt][r]{%
%        \raisebox{-2em}{%
%            \colorbox{white}{\includegraphics[width=.8\linewidth]{invoice-001.pdf}}%
%        }\hspace*{1em}%
%    }%
%    \caption{Factuur 1 en 2}
%\end{figure}
