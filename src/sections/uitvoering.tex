\section{Execution}\label{sec:output}
% Advanced YAML payload examples, + sequence diagram command line
Now that the legacy invoice has been fully transformed, let's see what the result looks like.
% explain jobname

\subsection{The Template Version}
Without providing any values, we get the following result, as shown in Figure~\ref{fig:template}.\\
\setlength\fboxsep{0pt}%
\begin{figure}[!ht]%
    \fbox{\includegraphics[width=\linewidth-1pt]{invoice-template.pdf}}%
    \caption{\texttt{invoice-template.pdf}}\label{fig:template}%
\end{figure}
As mentioned earlier, \pkg{lua-placeholders} can only be compiled with \LuaLaTeX.
Additionally, the option \texttt{--shell-escape} is required to load YAML files.
The example can be compiled as follows:
\begin{lstlisting}[language=bash,caption={Compiling with \texttt{lualatex}}]
lualatex --shell-escape \
     --jobname=invoice-template \
     --output-directory=$(OUTPUT_DIR) \
     invoice
\end{lstlisting}
Where \texttt{\$(OUTPUT\_DIR)} is the desired output directory and \texttt{invoice-template} not corresponding to any of the payload files, therefore making it an invoice template.

However, if you are drafting a template, continually generating with \texttt{latexmk} is more user-friendly:
\begin{lstlisting}[language=bash,caption={Compiling with \texttt{latexmk}}]
latexmk -pvc -lualatex \
    --shell-escape \
    --jobname=invoice-template \
    --output-directory=$(OUTPUT_DIR) \
    invoice
\end{lstlisting}
This way, you don't have to recompile every time there's a change in \TeX, it happens automatically.
It's also very useful on the application level for previewing the result while filling in a form.

\subsection{YAML Values}
To get a filled-in invoice, we will need the following YAML files:

\dirtree{%
    .1 \meta{project dir}.
    .2 invoices.
    .3 \meta{invoice}.yaml.
    .2 suppliers.
    .3 \meta{supplier}.yaml.
    .2 styles.
    .3 \meta{style}.yaml.
    .2 clients.
    .3 \meta{client}.yaml.
}

This structure is based on the implementation described in Chapter~\ref{sec:preamble}.
Before discussing the contents of the YAML files, let's first consider alternative project structures.

\subsubsection{Alternative Project Structure}
Everyone is free to create their desired folder structure.
For example, you could place styles under\\
\hspace*{4em}\texttt{/suppliers/\meta{supplier}/style.yaml}\\
so that you can even omit the \texttt{style} field in the supplier recipe.
Another good example is placing the \texttt{clients} directory under the supplier level, so you don't accidentally mix clients of suppliers.
This could be achieved by doing the following:
\dirtree{%
    .1 \meta{project dir}.
    .2 suppliers.
    .3 \meta{supplier}.yaml.
    .3 \meta{supplier}.
    .4 \meta{client}.yaml.
}
\noindent
This way, the implementation for loading clients would require the variables \meta{supplier} and \meta{client}, and then proceed to the path\\
\hspace*{4em}\texttt{suppliers/\meta{supplier}/\meta{client}.yaml}.

The same consideration could be applied to the invoices, however, this is a more complex scenario, as the invoice data is based on the \cs{jobname} in the implementation of Chapter~\ref{sec:preamble}.
One possible solution for this is to manage the project per supplier.
You can then place the \textit{recipes} in the \texttt{\$TEXMFHOME/tex} directory so that they are present for all projects.
Here's an example of a possible project structure:\\
\begin{minipage}{.49\linewidth}
    \dirtree{%
        .1 \$HOME/texmf/tex.
        .2 recipes.
        .3 invoice.yaml.
        .3 client.yaml.
        .3 supplier.yaml.
        .3 style.yaml.
        .2 invoice.cls.
        .2 invoice.tex.
    }
\end{minipage}%
\hfill
\begin{minipage}{.49\linewidth}
    \dirtree{%
        .1 \meta{project dir}.
        .2 invoices.
        .3 \meta{invoice}.yaml.
        .2 clients.
        .3 \meta{client}.yaml.
        .2 supplier.yaml.
        .2 style.yaml.
    }
\end{minipage}
In this example, all data related to a supplier is separated, including the client information and the final invoices.

\subsection{Suppliers and Clients}\label{sec:supplier and client}
In the example result of GinVoice, a client named \textit{Juicing Joker} was shown.
In YAML, this would translate to:
\lstinputlisting[captionpos=b,caption={\texttt{clients/juicing-joker.yaml}}]{demo/clients/juicing-joker.yaml}

This way, the client can be referenced in the invoice with \texttt{juicing-joker}.

For the supplier, we saw \textit{Grapefruit Inc.}\ in the example, which translates to:
\lstinputlisting[captionpos=b,caption={\texttt{suppliers/grapefruit.yaml} or \texttt{grapefruit/supplier.yaml}}]{demo/suppliers/grapefruit.yaml}

\onecolumn
\begin{figure}[!ht]
%    \centering
    \begin{subfigure}{.32\linewidth}
        \fbox{\includegraphics[width=\linewidth-1pt]{invoice2-template.pdf}}
        \caption{\texttt{invoice-template.pdf}}\label{fig:result alt}
    \end{subfigure}\hfill
    \begin{subfigure}{.32\linewidth}
        \fbox{\includegraphics[width=\linewidth-1pt]{invoice-001.pdf}}
        \caption{\texttt{invoice-001.pdf}}\label{fig:result 1}
    \end{subfigure}\hfill
    \begin{subfigure}{.32\linewidth}
        \fbox{\includegraphics[width=\linewidth-1pt]{invoice-002.pdf}}
        \caption{\texttt{invoice-002.pdf}}\label{fig:result 2}
    \end{subfigure}
    \caption{Invoice Examples}\label{fig:pdfs}
\end{figure}
\begin{multicols}{2}
    And for the style:
    \lstinputlisting[captionpos=b,caption={\texttt{styles/grapefruit.yaml} or \texttt{grapefruit/style.yaml}}]{demo/styles/grapefruit.yaml}

    The advantage of the alternative project structure is that \texttt{invoice-template} automatically picks up the styling as well as the supplier information, as seen in Figure~\ref{fig:result alt}.

    \subsection{Invoices}
    To create an invoice that exactly matches the standard example of GinVoice, as shown in Figure~\ref{fig:result 1}, we use the following YAML example:
    \lstinputlisting[xleftmargin=15pt,numbers=left,linerange=-15,caption={\texttt{invoices/invoice-001.yaml}}]{demo/invoices/invoice-001.yaml}
    The actors \texttt{grapefruit} and \texttt{juicing-joker}, discussed in Chapter~\ref{sec:supplier and client}, are reflected in the invoice.
    With other values it could make the difference between figure~\ref{fig:result 1} and~\ref{fig:result 2}.
    Additionally, the example has the same general information to achieve the same result.
    In the \texttt{records} field, you can see that one row of the table takes up many lines.
    In the second row of the table, you can see that the \texttt{description} field has an empty value.
%    \columnbreak
    If the quotes are omitted in YAML, you'll get an error when converting to data.
    Since the rows don't differ too much from each other, we continue the example at the \texttt{totals} field:
    \lstinputlisting[xleftmargin=15pt,numbers=left,linerange=34-40,firstnumber=34]{demo/invoices/invoice-001.yaml}
%    The only difference between compiling the template version is the \cs{jobname}.
\end{multicols}
\twocolumn
\lstinputlisting[xleftmargin=15pt,numbers=left,linerange=41-,firstnumber=41]{demo/invoices/invoice-001.yaml}
Finally, in the example, we see the totals and the closing text.

This invoice can then be compiled with the following command:
\begin{lstlisting}[language=bash]
lualatex --shell-escape \
    --jobname=invoice-001 \
    --output-directory=$(OUTPUT_DIR) \
      invoice
\end{lstlisting}
Where both \texttt{invoice-001} and \texttt{invoice-002} correspond to actual YAML files.

%\begin{figure}[!ht]
%    \centering
%    \includegraphics[width=.8\linewidth]{invoice-002.pdf}%
%    \makebox[0pt][r]{%
%        \raisebox{-2em}{%
%            \colorbox{white}{\includegraphics[width=.8\linewidth]{invoice-001.pdf}}%
%        }\hspace*{1em}%
%    }%
%    \caption{Factuur 1 en 2}
%\end{figure}
