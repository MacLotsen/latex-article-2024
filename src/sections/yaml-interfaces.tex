\section{YAML-Interfaces met \texttt{lua-placeholders}}
Dit hoofdstuk introduceert het gebruik van \pkg{lua-placeholders} in YAML-interfaces voor het verbeteren van factuur\-template-functionaliteiten.
We zullen bespreken waarom het upgraden van bestaande factuurtemplates naar een meer flexibele en geautomatiseerde oplossing noodzakelijk is.
Daarnaast zullen we de voordelen van \pkg{lua-placeholders} introduceren en hun toepassing in YAML-interfaces verkennen.
Het uiteindelijke doel is om een efficiënte en aanpasbare facturatie-oplossing te bieden, die gemakkelijk geïntegreerd kan worden in \LaTeX- en andere documentautomatisering-systemen.

\subsection{YAML-specificaties}
We zullen nu de YAML-specificaties m.b.t.\ de factuur bekijken:

\subsubsection{De Factuur}
\lstinputlisting[language=YAML,caption={\texttt{recipes/invoice.yaml}},numbers=left,xleftmargin=15pt]{demo/recipes/invoice.yaml}
De YAML-specificatie geeft een overzicht van alle variabelen en hun eigenschappen, inclusief bijzondere situaties zoals het gebruik van LaTeX-code in de standaardwaarde van de titel en de complexe placeholders in het bericht.
Verder worden de waarden van de variabelen supplier en client gebruikt om de waarden van andere onderdelen in te laden, maar de configuratie hiervan wordt in het volgende hoofdstuk besproken.

\subsubsection{Klant}
\lstinputlisting[language=YAML,caption={\texttt{recipes/client.yaml}},numbers=left,xleftmargin=15pt]{demo/recipes/client.yaml}

\subsubsection{Leverancier}
\lstinputlisting[language=YAML,caption={\texttt{recipes/supplier.yaml}},numbers=left,xleftmargin=15pt]{demo/recipes/supplier.yaml}
In het geval van de YAML-specificatie voor de leverancier heeft het veld \texttt{style} dezelfde functie als \texttt{supplier} en \texttt{client} van de factuur, waardoor de gebruiker kan kiezen welke stijl wordt toegepast.

Een ander interessant veld in deze specificatie is \texttt{extra fields}.
Dit veld maakt gebruik van het type \texttt{table} om extra informatievelden toe te staan, zoals bijvoorbeeld het rekeningnummer van de leverancier, het BTW-nummer, of andere relevante details.
Het gebruik van een tabel in plaats van een vast aantal velden geeft de eindgebruiker de flexibiliteit om zoveel extra informatie toe te voegen als nodig is, zonder beperkingen.

\subsubsection{Stijl}
\lstinputlisting[language=YAML,caption={\texttt{recipes/style.yaml}},numbers=left,xleftmargin=15pt]{demo/recipes/style.yaml}
Deze YAML-specificatie definieert de stijlopties die van toepassing zijn op de factuur.

\subsection{Nieuwe \LaTeX-template}
Nu de YAML-specificaties in orde zijn kunnen we over stappen naar het integreren in \LaTeX.

\subsubsection{Specificaties Inladen}
YAML-specificaties, ook wel \textit{recipes} genoemd, worden ingeladen met \cs{loadrecipe}.
\lstinputlisting[language={[LaTeX]TeX},numbers=left,xleftmargin=15pt,firstnumber=44,linerange=44-47]{demo/invoice.tex}
Je ziet voor de \texttt{invoice} specificatie dat het de namespace \cs{jobname} krijgt.
Dit heeft te maken met het feit dat macro \cs{param} standaard \cs{jobname} als \meta{namespace} gebruikt, wat het gebruik versimpelt.

De andere \textit{recipes} geven geef extra \meta{namespace} op, wat betekent dat ze de `basename' van het pad als namespace dragen.
In dit geval respectievelijk \texttt{supplier}, \texttt{client} en \texttt{style}.

\subsubsection{Munteenheid}
Voor wat betreft de munteenheid heb ik gekozen het in de macro \cs{currency} te vermommen.
Dit vanwege het feit dat het ook in andere bestanden gebruikt wordt, zoals \texttt{invoice.cls}.
\lstinputlisting[language={[LaTeX]TeX},numbers=left,xleftmargin=15pt,firstnumber=49,linerange=49]{demo/invoice.tex}
Mocht de \meta{currency} niet gezet zijn, dan wordt de standaardwaarde van \texttt{style.yaml} gebruikt, namelijk \cs{EUR}.

\subsubsection{Waarden Inladen}
\lstinputlisting[language={[LaTeX]TeX},numbers=left,xleftmargin=15pt,firstnumber=51,linerange=51-54]{demo/invoice.tex}

\lstinputlisting[language={[LaTeX]TeX},numbers=left,xleftmargin=15pt,firstnumber=56,linerange=56-58]{demo/invoice.tex}

\lstinputlisting[language={[LaTeX]TeX},numbers=left,xleftmargin=15pt,firstnumber=60,linerange=60-71]{demo/invoice.tex}

\subsection{Verwerken in Document}

\subsubsection{Header (Object)}
Wat eerder neer kwam op \cs{makeheader}, gaat helaas nu niet meer op.
Een bijzonder hinderlijke keuze was het, om het gegenereerde bestand \texttt{meta.tex} vanuit \texttt{invoice.cls} in te laden.
Voor de nieuwe versie is dit ongewenst, en ik raad het je ook zeker \underline{niet} aan om data in documentklassen te verwerken.

Voor nu gaan we er van uit dat \texttt{invoice.cls} zodanig is aangepast dat het geen fouten meer veroorzaakt en dat \cs{makeheader} nu de titel en subtitel als argumenten verwacht:
\begin{lstlisting}[language={[LaTeX]TeX}]
\thispagestyle{headermain}
\makeheader
    {\param{title}}
    {\param{subtitle}}
\vspace{2cm}
\end{lstlisting}
In het voorbeeld zijn er maar twee verschillen met betrekking tot de nieuwe versie, namelijk:\\
\cs{title} \textrightarrow \lstinline[language={[LaTeX]TeX}]|\param{title}|\\
\cs{subtitle} \textrightarrow \lstinline[language={[LaTeX]TeX}]|\param{subtitle}|

\subsubsection{Informatie}

\subsubsection{Tabel}

\subsubsection{Slottekst}
