\section{Discussie}

\subsection{\LaTeX\ Compilers}
In het artikel ga ik uit van de \LuaLaTeX-compiler.
Voor \LaTeX-gebruikers die andere compilers gebruiken, biedt \pkg{lua-placeholders} geen oplossing.
Hoewel sommige compilers alsnog ondersteuning bieden voor Lua, houdt \pkg{lua-placeholders} hier geen rekening mee.
Onderzoek en implementatie zou het draagvlak van \pkg{lua-placeholders} voor de \LaTeX-gemeenschap kunnen verbeteren.

\subsection{JSON vs YAML}
Op de keuze van YAML in plaats van JSON ben ik niet ingegaan in het artikel. Beide zijn bedoeld voor data en
hoewel JSON bekender is en meer compatibiliteit heeft met programmeertalen, heb ik juist voor YAML gekozen in belang van de leesbaarheid van \LaTeX-broncode.
Zoals uitvoerig gedemonstreerd, bevatten de bestanden veel \LaTeX-broncode.
Bij gebruik van JSON zou iedere backslash ge-\textit{escaped} moet worden.
% \begin{lstlisting}[style=yaml,caption={YAML-voorbeeld}]
% title: Factuur \param{number}
% \end{lstlisting}
% \begin{lstlisting}[style=json,caption={JSON-voorbeeld}]
% {
%     "title": "\\param{number}"
% }
% \end{lstlisting}

Vergelijk bijvoorbeeld:
\begin{verbatim}
title: Factuur \param{number}
\end{verbatim}
en
\begin{verbatim}
{
  "title": "\\param{number}"
}
\end{verbatim}

Voor een \LaTeX-gebruiker lijkt het mij prettiger om waarden aan te passen in YAML voor testdoeleinden, dan in JSON\@.

\newpage
\subsection{Roadmap GinVoice}
De ontwikkeling heeft enige tijd stil gestaan, echter ben ik kort geleden erachter gekomen dat de oplossing ook voor Windows platformen kan werken.
Het brengen van GinVoice naar het Windows platform verbreedt de doelgroep aanzienlijk en zou naar mijn verwachting meer draagvlak voor \LaTeX\ kunnen opleveren.

Voor wat betreft het introduceren van \pkg{lua-placeholders} zijn er nog een paar obstakels te overbruggen, zoals uitdagingen met betrekking tot vertaling en de niet besproken variabele kolomdefinitie, wat nu juist een gebruiksvriendelijk onderdeel van de applicatie is.
