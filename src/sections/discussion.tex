\section{Discussion}

\subsection{\LaTeX\ Compilers}
In the article, I assume the \LuaLaTeX\ compiler.
For \LaTeX\ users using other compilers, \pkg{lua-placeholders} does not provide a solution.
Although some compilers still offer support for Lua, \pkg{lua-placeholders} does not take this into account.
Research and implementation could improve the adoption of \pkg{lua-placeholders} within the \LaTeX\ community.

\subsection{JSON vs YAML}
I did not delve into the choice of YAML over JSON in the article.
Both are intended for data, and while JSON is more well-known and has broader compatibility with programming languages, I chose YAML for the sake of readability of \LaTeX\ source code.
As demonstrated extensively, the files contain a lot of \LaTeX\ source code.
When using JSON, every backslash would need to be escaped.
For example:
\begin{lstlisting}[style=yaml,caption={YAML example}]
title: Invoice \param{number}
\end{lstlisting}
\begin{lstlisting}[style=json,caption={JSON example}]
{
    "title": "\\param{number}"
}
\end{lstlisting}
For a \LaTeX\ user, I find it more convenient to adjust values in YAML for testing purposes than in JSON.

\subsection{GinVoice Roadmap}
Development has been stagnant for some time, but I recently discovered that the solution can also work for Windows platforms.
Bringing GinVoice to the Windows platform significantly expands the target audience and, in my expectation, could garner more support for \LaTeX.

Regarding the introduction of \pkg{lua-placeholders}, there are still a few obstacles to overcome, such as challenges related to translation and the variable column definition, which is precisely a user-friendly part of the application that has not been discussed.
