\section{Conclusion}\label{sec:conclusion}

In this article, we have explored how YAML interfaces can be used as a flexible and customizable way to manage and integrate invoice templates in \LaTeX.
By utilizing YAML files for specifying \textit{recipes} and invoice data, we have presented a structured approach for generating invoices using \LuaLaTeX\ and the \pkg{lua-placeholders} package.

Our approach enables easy customization and management of invoice templates, providing users with complete control over the layout and content of the invoices.
Furthermore, we have considered alternative project structures that can further enhance the flexibility and management of invoice templates.

The \pkg{lua-placeholders} package has played a crucial role in optimizing the document interface by providing a powerful and flexible mechanism for replacing variables in \LaTeX\ documents.
By leveraging \LuaLaTeX, users can benefit from the speed and efficiency of Lua for data processing while still enjoying the typographic precision of \LaTeX\ for generating professionally looking documents.

For applications like GinVoice, this approach offers significant benefits as users can now more easily create and execute \LaTeX\ invoice templates and potentially integrate them with the application interface.

We encourage our readers to explore and implement \pkg{lua-placeholders} in their own projects.
The possibilities for system integration and document interfacing with \pkg{lua-placeholders} are extensive, providing users with the flexibility to manage complex document workflows with minimal effort.
By using \pkg{lua-placeholders}, users can enhance the efficiency of their document processes while maintaining the quality and consistency of their documents.

It is also important to consider some considerations for others considering using this package.
While \pkg{lua-placeholders} is a powerful tool for document interfacing, it requires a certain level of familiarity with YAML and \LaTeX.
In the case of the alternative project structure, basic knowledge of the TDS (\TeX\ Directory Structure) is also required.

Finally, we have compared with alternative methods and determined that \pkg{lua-placeholders} is the best choice due to its flexibility, speed, and integration capabilities with \LaTeX.
While there are other approaches to generating documents using \LaTeX, \pkg{lua-placeholders} offers a unique combination of functionality and ease of use, making it an attractive option for document interfacing.

Overall, our research provides a practical and efficient solution for managing and integrating invoice templates in \LaTeX\ projects, allowing users to benefit from a seamless workflow for invoicing and reporting.
