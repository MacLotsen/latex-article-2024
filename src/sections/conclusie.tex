\section{Conclusion}\label{sec:conclusie}
This article has presented a structured approach for generating invoices using \LaTeX, with a focus on YAML interfaces and the \pkg{lua-placeholders} package.

For end-users, our approach offers an enhanced experience through the optimized version of GinVoice.
By properly separating application development and \LaTeX\ typesetting, end-users can benefit from an optimal experience with multiple templates to choose from while seamlessly working with GinVoice.

For \LaTeX\ users, it is now easier to contribute to the templates for the new GinVoice application.
They can effortlessly generate invoice templates using only \LaTeX, providing them with the freedom to customize and contribute templates without the need for knowledge of the internal workings of the application.

Programmers benefit from the use of \pkg{lua-placeholders} as they no longer need to have knowledge of \LaTeX, which has a steep learning curve.
Instead, they can focus on developing the application and integrating \LaTeX\ templates, streamlining development processes and increasing efficiency.

Thus, the \pkg{lua-placeholders} package has addressed all stakeholders by introducing a highly accessible file format, namely YAML.
This allows for the optimization of invoicing processes and streamlining document workflows, saving time and effort in generating invoices and reports.

I encourage readers to explore and implement the proposed approach in their own projects.
By leveraging YAML and \pkg{lua-placeholders}, users can benefit from efficient and professional document processes, enabling them to achieve a seamless workflow for invoicing and reporting.
