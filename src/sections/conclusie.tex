\section{Conclusie}\label{sec:conclusie}
Dit artikel heeft een gestructureerde aanpak gepresenteerd voor het genereren van facturen met behulp van \LaTeX, waarbij YAML-interfaces en het \pkg{lua-placeholders} pakket centraal staan.

Voor eindgebruikers biedt onze benadering een verbeterde ervaring via de geoptimaliseerde versie van GinVoice.
Door de juiste scheiding tussen applicatieontwikkeling en \LaTeX-typesetting kunnen eindgebruikers profiteren van een optimale ervaring met meerdere sjablonen om uit te kiezen, terwijl ze toch naadloos kunnen werken met GinVoice.

Voor \LaTeX-gebruikers is het nu eenvoudiger om bij te dragen aan de sjablonen voor de nieuwe GinVoice-applicatie.
Ze kunnen moeiteloos factuursjablonen genereren met alleen \LaTeX, waardoor ze de vrijheid hebben om sjablonen aan te passen en bij te dragen zonder de noodzaak van kennis over de interne werking van de applicatie.

Programmeurs profiteren van het gebruik van \pkg{lua-placeholders}, omdat ze geen kennis meer nodig hebben van \LaTeX, wat een grote leercurve heeft.
In plaats daarvan kunnen ze zich concentreren op het ontwikkelen van de applicatie en het integreren van \LaTeX-sjablonen, waardoor de ontwikkelingsprocessen worden gestroomlijnd en de efficiëntie wordt verhoogd.

Het pakket \pkg{lua-placeholders} heeft dus met alle actoren rekening gehouden door het introduceren van een zeer toegankelijk bestandsformaat, namelijk YAML\@.
Dit maakt het mogelijk om facturatieprocessen te optimaliseren en de documentworkflow te stroomlijnen, waardoor tijd en moeite worden bespaard bij het genereren van facturen en rapporten.

We moedigen onze lezers aan om de voorgestelde aanpak te verkennen en te implementeren in hun eigen projecten.
Door gebruik te maken van YAML en \pkg{lua-placeholders}, kunnen gebruikers profiteren van efficiënte en professionele documentprocessen, waardoor ze een naadloze workflow voor facturatie en rapportage kunnen realiseren.
