\section{Conclusie}\label{sec:conclusie}
In dit artikel hebben we niet alleen de implementatie van factuursjablonen in GinVoice onderzocht, maar ook een innovatieve methode voorgesteld om deze sjablonen naadloos te integreren met het \LaTeX-ecosysteem.
Door YAML te gebruiken als tussenlaag en \pkg{lua-placeholders} voor dynamische invoegingen, hebben we niet alleen een robuuste en flexibele oplossing geboden voor het genereren van facturen, maar ook een raamwerk gecreëerd waarin verschillende documentonderdelen, zoals klantinformatie, document-overkoepelend kunnen worden ingezet.

Deze aanpak biedt \LaTeX-gebruikers niet alleen de vrijheid om factuursjablonen naar wens aan te passen, maar opent ook de deur naar een breder scala aan toepassingen.
Door dezelfde YAML-gebaseerde structuur te gebruiken, kunnen verschillende documenten, zoals contracten en facturen, met gemak worden gegenereerd en onderhouden.
Dit vergroot niet alleen de consistentie tussen verschillende documenttypen, maar verhoogt ook de efficiëntie van het documentatieproces als geheel.

Het gebruik van \pkg{lua-placeholders} in combinatie met YAML maakt het mogelijk om dynamische content toe te voegen aan sjablonen, waardoor gebruikers een meer gestroomlijnde workflow ervaren.
Deze flexibiliteit maakt het gemakkelijk om gegevens en opmaak van verschillende documenten te scheiden, terwijl het tegelijkertijd de mogelijkheid biedt om deze onderdelen document-overkoepelend in te zetten.

Als conclusie biedt deze aanpak niet alleen een waardevolle bijdrage aan het optimaliseren van factureringsprocessen, maar opent het ook nieuwe mogelijkheden voor het efficiënt genereren en beheren van verschillende soorten documenten binnen een organisatie.
