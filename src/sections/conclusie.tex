\section{Conclusion}\label{sec:conclusion}
In this study, we have not only examined the implementation of invoice templates in GinVoice but also proposed an innovative method to seamlessly integrate these templates with the \LaTeX\ ecosystem.
By using YAML as an intermediate layer and \pkg{lua-placeholders} for dynamic insertions, we have provided a robust and flexible solution for invoice generation while creating a framework where various document components, such as client information, can be utilized across documents.

This approach not only grants \LaTeX\ users the freedom to customize invoice templates as desired but also opens the door to a wider range of applications.
By employing the same YAML-based structure, different documents, including contracts and invoices, can be generated and maintained with ease.
This not only enhances consistency across various document types but also boosts the efficiency of the documentation process as a whole.

The utilization of \pkg{lua-placeholders} in conjunction with YAML enables the addition of dynamic content to templates, resulting in a more streamlined workflow for users.
This flexibility makes it easy to separate data and formatting across different documents while allowing these components to be used across documents.

In conclusion, this approach not only makes a valuable contribution to optimizing billing processes but also unveils new possibilities for efficiently generating and managing various types of documents within an organization.
